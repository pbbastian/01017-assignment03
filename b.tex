%!TEX root = document.tex
\section*{Opgave B}

\subsection*{(a)}
Jeg ser på følgende formel:
\begin{equation}
  \left( \left( p \rightarrow r \right) \lor \left( q \rightarrow r \right) \right) \rightarrow \left( \left( p \lor q \right) \rightarrow r \right)
\end{equation}
Jeg vil benytte tableau-metoden til at afgøre om formlen er gyldig:
\begin{equation*}
  \begin{tikzpicture}[level distance=8mm,sibling distance=50mm]\tikzstyle{level 2}=[sibling distance=25mm]
    \node{$\left( \left( p \rightarrow r \right) \lor \left( q \rightarrow r \right) \right) \rightarrow \left( \left( p \lor q \right) \rightarrow r \right) : \T\ \checkmark$}
    child {
      node {$\left( p \rightarrow r \right) \lor \left( q \rightarrow r \right) : \F\ \checkmark$}
      child {
        node {$p \rightarrow r : \F\ \checkmark$}
        child {
          node {$q \rightarrow r : \F\ \checkmark$}
          child {
            node {$p : \T$}
            child {
              node {$r : \F$}
              child {
                node {$q : \T$}
                child {
                  node {$r : \F$}
                  \open
                }
              }
            }
          }
        }
      }
    }
    child {
      node {$\left( p \lor q \right) \rightarrow r : \T\ \checkmark$}
      child {
        node {$p \lor q : \F\ \checkmark$}
        child {
          node {$p : \F$}
          child {
            node {$q : \F$}
            \open
          }
        }
      }
      child {
        node {$r : \T$}
        \open
      }
    };
  \end{tikzpicture}
\end{equation*}
Det ses at der ikke er nogle lukkede grene og derved ikke nogen måde hvorpå formlen kan blive falsk. Altså er formlen gyldig.

\subsection*{(b)}
Jeg ser på følgende formel:
\begin{equation}
  p \rightarrow \left( \left( q \rightarrow r \right) \rightarrow \left( \left( p \rightarrow q \right) \rightarrow \left( p \rightarrow r \right) \right) \right)
\end{equation}
Jeg vil benytte tableau-metoden til at afgøre om formlen er gyldig:
\begin{equation*}
  \begin{tikzpicture}[level distance=8mm,sibling distance=50mm]\tikzstyle{level 2}=[sibling distance=25mm]
    \node{$p \rightarrow \left( \left( q \rightarrow r \right) \rightarrow \left( \left( p \rightarrow q \right) \rightarrow \left( p \rightarrow r \right) \right) \right) : \F\ \checkmark$}
    child {
      node {$p : \T$}
      child {
        node {$\left( q \rightarrow r \right) \rightarrow \left( \left( p \rightarrow q \right) \rightarrow \left( p \rightarrow r \right) \right) : \F\ \checkmark$}
        child {
          node {$q \rightarrow r : \T\ \checkmark$}
          child {
            node {$\left( p \rightarrow q \right) \rightarrow \left( p \rightarrow r \right) : \F\ \checkmark$}
            child {
              node {$p \rightarrow q : \T\ \checkmark$}
              child {
                node {$p \rightarrow r : \F\ \checkmark$}
                child {
                  node {$p : \T$}
                  child {
                    node {$r : \F$}
                    child {
                      node {$q : \F$}
                      child {
                        node {$p : \F$}
                        \closed
                      }
                      child {
                        node {$q : \T$}
                        \closed
                      }
                    }
                    child {
                      node {$r : \T$}
                      \closed
                    }
                  }
                }
              }
            }
          }
        }
      }
    };
  \end{tikzpicture}
\end{equation*}
Det ses at det kun er lukkede grene og derved ingen måde hvorpå formlen kan blive falsk. Altså er formlen gyldig.