%!TEX root = document.tex
\section*{Opgave D}
Jeg benytter følgende propositionsvariable:
\begin{itemize}
  \item $R_t$: René flytter sit tårn
  \item $P_d$: Per flytter sin dronning
  \item $P_l$: Per flytter sin løber
  \item $R_s$: René smadrer skakbrættet
\end{itemize}
Dem vil jeg benytte til at omskrive de 4 sætninger til logiske udtryk. Jeg ser først på den første sætning, som er:
\begin{quote}
  \emph{René vil kun flytte sin tårn, hvis Per flytter sin dronning.}
\end{quote}
Altså må der være tale om en implikation. Det kan udtrykkes ved:
\begin{equation}
  P_d \rightarrow R_t
\end{equation}
Anden sætning er:
\begin{quote}
  \emph{Hvis Per flytter sin løber, vil René smadre skakbrættet.}
\end{quote}
Igen må der være tale om en implikation. Det kan udtrykkes ved:
\begin{equation}
  P_l \rightarrow R_s
\end{equation}
Tredje sætning er:
\begin{quote}
  \emph{Det er ikke tilfældet, at Per vil flytte sin dronning og ikke flytte sin løber.}
\end{quote}
Der må være tale om en negering af en implikation. Det kan udtrykkes ved:
\begin{equation}
  \lnot \left( P_d \rightarrow \lnot P_l \right)
\end{equation}
Fjerde sætning er:
\begin{quote}
  \emph{Hvis Per flytter sin dronning, vil René flytte sit tårn og smadre skakbrættet.}
\end{quote}
Der må være tale om to ting, der sker hvis én ting sker. Det kan udtrykkes ved:
\begin{equation}
  \label{D4}
  P_d \rightarrow R_t \land R_s
\end{equation}
Samles de fire sætninger på samme måde som givet i opgaven fås:
\begin{equation}
  \begin{split}
    & P_d \rightarrow R_t\\
    & P_l \rightarrow R_s\\
    & \lnot \left( P_d \rightarrow \lnot P_l \right)\\
    \hline
    & P_d \rightarrow R_t \land R_s
  \end{split}
\end{equation}
Den logiske konsekvens kan opdeles i følgende to dele:
\begin{equation}
  P_d \rightarrow R_t \quad,\quad P_d \rightarrow R_s
\end{equation}
Den første del opfyldes af den første sætning, men den anden del kan ikke konkluderes ud fra de tre ligninger. Altså må slutningen være falsk.