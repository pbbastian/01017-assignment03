%!TEX root = document.tex
\section*{Opgave C}
Jeg vil bevise følgende lemma gælder:
\begin{description}
  \item[Lemma 1] Formlerne A og B er logisk ækvivalente hvis og kun hvis formlen $A \leftrightarrow B$ er gyldig.
\end{description}
Da lemmaet siger \emph{hvis og kun hvis} må det være det samme som at følgende to sætninger gælder:
\begin{description}
  \item[Sætning 1] Formlerne A og B er logisk ækvivalente hvis formlen $A \leftrightarrow B$ er gyldig.
  \item[Sætning 2] Formlen $A \leftrightarrow B$ er gyldig hvis formlerne A og B er logisk ækvivalente.
\end{description}
Jeg starter med at vise at sætning 1 gælder, hvor det antages at $A \leftrightarrow B$ er gyldig. Jeg kan opstille en sandhedstabel for $A \leftrightarrow B$:
  \begin{center}
    \begin{tabular}{cc|c}
    \textbf{$A$} & \textbf{$B$} & \textbf{$A \leftrightarrow B$} \\

    \hline

    $\T$ & $\T$ & $\T$ \\
    $\T$ & $\F$ & $\F$ \\
    $\F$ & $\T$ & $\F$ \\
    $\F$ & $\F$ & $\T$
    \end{tabular}
  \end{center}
Det ses at hvis $A \leftrightarrow B$ skal være gyldig, er der kun to mulige tilfælde: Både $A$ og $B$ er sande eller eller både $A$ og $B$ er falske. Det vil sige at de skal have samme værdi for at gøre udtrykket gyldigt, hvilket jo betyder at de er logisk ækvivalente. Hermed er sætning 1 vist.\\

\noindent
Jeg vil nu se på sætning 2, hvor det antages at formlerne A og B er logisk ækvivalente. Givet det er der altså to mulige tilfælde: Både $A$ og $B$ er sande eller eller både $A$ og $B$ er falske. Hvis de værdier benyttes i $A \leftrightarrow B$ vil resultatet (jf. sandhedstabellen fra før) altid være sandt. Det må betyde at $A \leftrightarrow B$ er gyldigt, da der for de mulige værdier af $A$ og $B$ ikke er et tilfælde hvor $A \leftrightarrow B$ er falsk. Hermed er sætning 2 vist.\\

\noindent
Eftersom både sætning 1 og sætning 2 gælder, må lemmaet også gælde. Hermed er lemma 1 vist.